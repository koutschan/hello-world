\documentclass{sig-alternate}
\let\proof\undefined
\let\endproof\undefined
\usepackage{amsthm,amssymb,amsmath}
\usepackage{graphics}
\usepackage{bbm}
\usepackage{tikz}
\usepackage[plainpages=false,pdfpagelabels,colorlinks=true,citecolor=blue,hypertexnames=false]{hyperref}
\usepackage{color}
\overfullrule=1mm

\newtheorem{theorem}{Theorem}
\newtheorem{notation}[theorem]{Convention}
\newtheorem{prop}[theorem]{Proposition}
\newtheorem{corollary}[theorem]{Corollary}
\newtheorem{lemma}[theorem]{Lemma}
\newtheorem{remark}[theorem]{Remark}
\newtheorem{algorithm}[theorem]{Algorithm}
\newtheorem{problem}[theorem]{Problem}
\newtheorem{defi}[theorem]{Definition}
\newtheorem{example}[theorem]{Example}
\newtheorem{fact}[theorem]{Fact}
\def\qed{\quad\rule{1ex}{1ex}}
\def\ord{\operatorname{ord}}
\def\<#1>{\langle#1\rangle}

\newcommand{\red}{\color{red}}
\newcommand{\blue}{\color{blue}}
\newcommand{\bQ}{ {\mathbb Q}}
\newcommand{\bA}{ {\mathbb A}}
\newcommand{\bE}{ {\mathbb E}}
\newcommand{\bB}{ {\mathbb B}}
\newcommand{\cB}{ {\mathcal B}}
\newcommand{\bC}{ {\mathbb C}}
\newcommand{\bF}{ {\mathbb F}}
\newcommand{\bN}{ {\mathbb N}}
\newcommand{\bZ}{ {\mathbb Z}}
\newcommand{\bK}{ {\mathbb K}}
\newcommand{\cM}{ {\mathcal M}}
\newcommand{\cN}{ {\mathcal N}}
\newcommand{\cS}{ {\mathcal S}}
\newcommand{\cO}{ {\mathcal O}}
\newcommand{\cE}{ {\mathcal E}}
\newcommand{\cP}{ {\mathcal P}}
\newcommand{\den } {{\rm den}}
\newcommand{\num }{{\rm num}}
\newcommand{\de } {\delta}
\newcommand{\pa}{\partial}
\newcommand{\spanning}{\text{span}}

\overfullrule=1ex

\newcommand{\ve} {{\bf e}}
\newcommand{\vp} {{\bf p}}
\newcommand{\vv} {{\bf v}}
\newcommand{\vq} {{\bf q }}
\newcommand{\vx} {{\bf x}}

\let\set\mathbbm
\def\lc{\operatorname{lc}}
\def\rank{\operatorname{rank}}
\def\lt{\operatorname{lt}}
\def\im{\operatorname{Im}}
\def\lclm{\operatorname{lclm}}

\begin{document}

\title{Reduction-Based Creative~Telescoping for \\ Fuchsian D-finite~Functions}

\numberofauthors{1}

\author{\medskip
Shaoshi Chen$^{1,2}$, \, Manuel Kauers$^{3}$, \, Christoph Koutschan$^{4}$ \\
\smallskip
       \affaddr{$^1$KLMM,\, AMSS, \,Chinese Academy of Sciences, Beijing, 100190, (China)}\\
       \smallskip
       \affaddr{$^2$SCG, Faculty of Mathematics, University of Waterloo, Ontario, N2L3G1, (Canada)}\\
              \smallskip
       \affaddr{$^3$Institute for Algebra, Johannes Kepler University, Altenberger Stra\ss e 69,
 A-4040 Linz, (Austria)}\\
        \smallskip
       \affaddr{$^4$RICAM, Austrian Academy of Sciences, Altenberger Stra\ss e 69, A-4040 Linz, (Austria)}\\
       \smallskip
      \email{schen@amss.ac.cn, manuel.kauers@jku.at}\\
      \email{christoph.koutschan@ricam.oeaw.ac.at}
}

\maketitle
%
\begin{abstract}
  Continuing a series of articles on the past few years on reduction based creative telescoping,
  we adapt Trager's Hermite reduction for algebraic functions to fuchsian D-finite functions and
  develop a reduction-based creative telescoping algorithm for these functions. 
\end{abstract}


\category{I.1.2}{Computing Methodologies}{Symbolic and Algebraic Manipulation}[Algebraic Algorithms]

\terms{Algorithms, Theory}

\keywords{Algebraic function, Integral basis, Trager's Reduction, Telescoper}

%\bigskip

\section{Introduction}\label{SECT:intro}

The classical question in symbolic integration is whether the integral of
a given function can be written in ``closed form''. In its most restricted form,
the question is whether for a given function~$f$ belonging to some domain $D$
there exists another function~$g$, also belonging to~$D$, such that $f=g'$. For
example, if $D$ is the field of rational functions, then for $f=1/x^2$ we can
find $g=-1/x$, while for $f=1/x$ no suitable $g$ exists. When no $g$ exists
in~$D$, there are several other questions we may ask. One possibility is to ask
whether there is some extension~$E$ of $D$ such that in $E$ there exists some
$g$ with $g'=f$. For example, in the case of elementary functions, Liouville's
principle restricts the possible extensions~$E$, and algorithms have been
designed to construct these extensions whenever possible. Another possibility is
to ask whether for some modification $\tilde f\in D$ of~$f$ there exists a $g\in
D$ such that $\tilde f=g'$. Creative telescoping is a question of this
type. Here we are dealing with domains~$D$ containing functions in several
variables, say $x$ and~$t$, and the question is whether there is a linear
differential operator~$P$, nonzero and free of~$x$, such that there exists a
$g\in D$ with $P\cdot f=g'$, where $g'$ denotes the derivative of $g$ with
respect to~$x$. Typically, $g$~itself has the form $Q\cdot f$ for some operator
$Q$ (which may be zero and need not be free of~$x$). In this case, we call $P$
a telescoper for~$f$, and $Q$ a certificate for~$P$.

Creative telescoping is the backbone of definite integration. Readers not
familiar with this technique are refered to the literature~\cite{PWZbook1996,Zeilberger1990c,Zeilberger1991,Zeilberger1990,Koepf1998}
for motivation, theory, algorithms, implementations, and applications. There are
several ways to find telescopers for a given $f\in D$. In recent years, an
approach has become popular which has the feature that it can find a telescoper
without also constructing the corresponding certificate. This is interesting
because certificates tend to be much larger than telescopers, and in some
applications only the telescoper is of interest. This approach was first
formulated for rational functions $f\in C(t,x)$ in~\cite{BCCL2010} and later
generalized to rational functions in several variables~\cite{bostan13}, to
hyperexponential functions~\cite{bostan13a} and, for the shift case, to hypergeometric
terms~\cite{chen15a} and binomial sums~\cite{bostan15}. In the present paper, we will extend
the approach to fuchsian D-finite functions.

The basic principle of the general approach is as follows. Assume that the
$x$-constants $\mathrm{Const}_x(D)=\{\,c\in D:c'=0\,\}$ form a field, i.e., that $D$
is a vector space over the field of $x$-constants. Assume further that there is
some $\mathrm{Const}_x(D)$-linear map $[\cdot]\colon D\to D$ such that for every
$f\in D$ there exists a $g\in D$ with $f-[f]=g'$. Such a map is called a
\emph{reduction.} For example, in $D=C(t,x)$ Hermite reduction produces for
every $f\in D$ some $g$ such that $f-g'$ is either zero or a rational function
with a square-free denominator. In this case, we can take $[f]=f-g'$.
In order to find a telescoper, we can compute $[f]$, $[\partial_t\cdot f]$, $[\partial_t^2\cdot f]$, \dots,
until we find that they are linearly dependent over $\mathrm{Const}_x(D)$.
When we find a relation
$p_0[f] + \cdots + p_r[\partial_t^r\cdot f] = 0$,
then, by linearity,
$[p_0 f + \cdots + p_r \partial_t^r\cdot f] = 0$,
and then, by definition of $[\cdot]$, there exists a $g\in D$ such that $(p_0+\cdots + p_r\partial_t^r)\cdot f=g'$.
In other words, $P=p_0+\cdots + p_r\partial_t^r$ is a telescoper.

There are two ways to guarantee that this method works. The first consists in
showing that $\{\,[f]:f\in D\,\}$ is a finite-dimensional vector space over
$\mathrm{Const}_x(D)$. This approach was taken in~\cite{BCCL2010,bostan13a}. It has the
nice additional feature that every bound for the dimension of this vector space
gives rise to a bound for the order of the telescoper. In particular, it implies
the existence of a telescoper. The second way requires that we already know for
other reasons that a telescoper exists. The idea is then to show that the
reduction $[\cdot]$ has the property that when $f\in D$ is such that there
exists a $g\in D$ with $g'=f$, then $[f]=0$. If this is the case and
$P=p_0+\cdots+p_r\partial_t^r$ is a telescoper for~$f$, then $P\cdot f$ is integrable
in~$D$, so $[P\cdot f]=0$, and by linearity $[f]$, \dots, $[\partial_t^r\cdot f]$ are
linearly dependent over $\mathrm{Const}_x(D)$. This means that the method won't
miss any telescoper. In particular, this argument has the nice feature that we
are guaranteed to find a telescoper of smallest possible order~$r$. This
approach was taken in~\cite{chen15a}.

Using an analog of Trager's Hermite reduction for algebraic
functions~\cite{trager84,geddes92}  adapted to fuchsian D-finite functions, we provide an
algorithm for finding the minimal order telescoper in Section~\ref{sec:hermite} using the
first argument. We then proceed to describe in Section~\ref{sec:polynomial} an additional
reduction which in combination with the Hermite reduction ensures that the
remainders live in a finite-dimensional vector space.
This gives a new proof of a bound for the order of the telescopers,
and in particular an independent proof for their existence.

\section{Fuchsian D-finite Functions}

Let $C$ be a field. We consider linear differential operators
$L=\ell_0+\cdots+\ell_n\partial_x^n$ with $\ell_0,\dots,\ell_n$ belonging to some ring
$R$ containing~$C$.
Typical choices for $R$ will be $C[x]$ or~$C(x)$.
When $\ell_n\neq0$, we say that $\ord(L):=n$ is the order of~$L$.

We write $R[\partial_x]$ for the algebra consisting of all these operators, together
with the usual addition and the unique non-commutative multiplication satisfying
$\partial_xc=c\partial_x$ for all $c\in C$ and $\partial_xx=x\partial_x+1$.
The algebra $R[\partial_x]$ acts on a differential $R$-module~$F$ via
\[
  (\ell_0+\ell_1\partial_x+\cdots+\ell_n\partial_x^n)\cdot f=
   \ell_0f + \ell_1f' + \cdots + \ell_n f^{(n)}.
\]
An element $y\in F$ is called a solution of an operator $L\in R[\partial_x]$ if
$L\cdot y=0$.

By $\bar C$ we denote some algebraically closed field containing~$C$ (not necessarily the smallest).
An operator $L$ of order~$n$ is called fuchsian at a point $a\in\bar C$ if
it admits $n$ linearly independent solutions in
\[
  C[[[x-a]]] := \bigcup_{\nu\in C} (x-a)^\nu\bar C[[x-a]][\log(x-a)].
\]
It is called fuchsian at $\infty$ if it admits $n$ linearly independent solutions in
\[
  C[[[x^{-1}]]] := \bigcup_{\nu\in C} x^{-\nu} \bar C[[x^{-1}]][\log(x)].
\]
It it simply called \emph{fuchsian} if it is fuchsian at all $a\in\bar C\cup\{\infty\}$.
Note that the exponents $\nu$ are restricted to~$C$, not to the larger field~$\bar C$.

For a fixed fuchsian operator $L$, we will consider the left $R[\partial_x]$-module
$A=R[\partial_x]/\<L>$, where $\<L>$ denotes the left ideal generated by~$L$ in
$R[\partial_x]$.  Then $1\in A$ is a solution of~$L$, because we have $L\cdot 1=L=0$
in~$A$. We can say that $A$ consists of all the ``functions'' $f$ which can be
obtained from a ``generic'' solution~$y$ of $L$ by applying some operator $P\in
R[\partial_x]$ to it. When $R$ is a field, then $A$ is an $R$-vector space of
dimension~$n=\ord(L)$, generated by $1,\partial_x,\dots,\partial_x^{n-1}$.

It is instructive to compare this setup to the situation for algebraic
functions. Comparing $A=R[\partial_x]/\<L>$ to an algebraic function field $R[Y]/\<M>$
(when $R$ is a field), our operator $L$ plays the role of the minimal
polynomial~$M$. In the algebraic case, $Y$~is a formal solution of the equation
$M=0$, similar as $1\in A$ is a formal solution of~$L$. Besides these formal
solutions there are, for each fixed $a\in\bar C$, exactly $\deg_Y(M)$ different
Puiseux series solutions of $M=0$ at places above~$a$. They correspond in the
differential setting to the series solutions of $L$ in $\bar C[[[x-a]]]$, which
generate a $\bar C$-vector space of dimension $\ord(L)$.

In the context of creative telescoping, we let $\bar C$ be some algebraically
closed field containing the rational function field $K=C(t)$, and we
use $R=K(x)$ instead of~$C(x)$. Integration will always be with respect to~$x$, but
besides the derivation $\partial_x$ there is now also the derivation with respect
to~$t$. The notation $f'$ will always refer to the derivative of $f$
with respect to~$x$. In addition to the operator algebra $R[\partial_x]$, we consider the
operator algebra $R[\partial_x,\partial_t]$, in which $\partial_x,\partial_t$
commute with each other (although they need not commute with elements of~$R$).

The action of $R[\partial_x]$ on $C[[[x-a]]]$ or $C[[[x^{-1}]]]$ is extended
to $R[\partial_t,\partial_x]$ by letting $\partial_t$ act coefficient-wise.
We further assume that the action of $R[\partial_x]$ on $A=R[\partial_x]/\<L>$ is
extended to an action of $R[\partial_t,\partial_x]$ on~$A$ in a way that is
compatible with the action of $R[\partial_t,\partial_x]$ on series domains.
This means that when $y\in C[[[x-a]]]$ is a solution of~$L$ and $f$ is an
element of~$A$, so that $f\cdot y$ is an element of $C[[[x-a]]]$, then we
want to have $(\partial_t\cdot f)\cdot y=\partial_t\cdot(f\cdot y)$, where
the $\cdot$ in $(\partial_t\cdot f)$ refers to the action of $R[\partial_t,\partial_x]$
on $A$ and the three other dots refer to the action on $C[[[x-a]]]$. 
If $u\in A$ is such that $\partial_t\cdot 1 = u$ and $U\in R[\partial_x]$ is such
that $u=U+\<L>$, then the annihilator $\mathfrak{a}=\mathrm{ann}(1)\subseteq R[\partial_t,\partial_x]$
of $1\in A$ in $R[\partial_t,\partial_x]$ contains $L$ and $\partial_t-U$. We
therefore have $\dim\mathfrak{a}=0$, and this is the usual
definition of D-finiteness in the case of several variables~\cite{Zeilberger1990,chyzak98,koutschan09,kauers14c}.

\section{Integral Bases}

Trager's Hermite reduction for algebraic functions rests on the notion of
integral bases. The notion of integral bases has been generalized to D-finite
functions last year~\cite{kauers15b}, and an algorithm was also given there for
computing such bases. We recall here the relevant definitions and properties.

Although the elements of a generalized series ring $C[[[x-a]]]$ are formal
objects, the series notation suggests certain analogies with complex
functions.  For simplicity, let us assume throughout that $C\subseteq\set
R$. Terms $(x-a)^\alpha\log(x-a)^\beta$ or $(\tfrac1x)^\alpha\log(x)^\beta$ are
called \emph{integral} if $\alpha>0$ or $\alpha=\beta=0$.
Integrality thus means that the corresponding complex
function does not diverge at the reference point~$a$ (the reference point is
$\infty$ in the case of $(\tfrac1x)^\alpha\log(x)^\beta$). A series in
$C[[[x-a]]]$ or $C[[[x^{-1}]]]$ is called integral if it only contains integral
terms. A non-integral series is said to have a \emph{pole} at the reference
point.  Note that in this terminology also $1/\sqrt{x}$ and $\log(x)$ have
``poles'' at~$0$, although the usual terminology reserves the word pole for
asymptotic behaviour which agrees with $x^n$ for some negative integer~$n$.

Note also that integrality at $a\in\bar C$ is not preserved by differentiation,
but if $f$ is integral at~$a$, then so is $(x-a)f'$. Somewhat conversely,
integrality at infinity is preserved by differentiation, we even have that
when $f$ is integral at infinity, then not only $f'$ but also $xf'$ is
integral at infinity.

Let $K$ be some field with $C\subseteq K\subseteq\bar C$.
Let $L\in K(x)[\partial_x]$ be a fuchsian operator. An element $f\in A=K(x)[\partial_x]/\<L>$
is called (locally) integral at $a\in\bar C\cup\{\infty\}$ if for every solution $y$
of $L$ in $C[[[x-a]]]$ or $C[[[x^{-1}]]]$, respectively, the series $f\cdot y$ is
integral. $f$~is called (globally) integral if it is locally integral at every
$a\in\bar C$ (``at all finite places''). Because of the Fuchs relation~\cite{ince26}, an
element $f\in A$ is integral at every $a\in\bar C\cup\{\infty\}$ if and only if it
is a constant (i.e., if $\partial_xf=0$ in~$A$).

The set of all integral elements $f\in A$ forms a $K[x]$-submodule of~$A$.
A basis $\{\omega_1,\dots,\omega_n\}$ of this module is called an \emph{integral basis}
for~$A$. In~\cite{kauers15b} an algorithm was given which computes an integral basis
for a given~$A$. This algorithm is a generalization of van Hoeij's
algorithm~\cite{vanHoeij94} for computing integral bases of algebraic function
fields.

For a fixed $a\in\bar C$, let $\bar C(x)_a$ be the ring of rational functions $p/q$
with $q(a)\neq0$, and write $\bar C(x)_\infty$ for the ring of all
rational functions $p/q$ with $\deg_x(p)\leq\deg_x(q)$.
Then the set of all $f\in A$ which are locally integral at some
fixed $a\in\bar C\cup\{\infty\}$ forms a $\bar C(x)_a$-module. A basis of this module is
called a \emph{local integral basis} at $a$ for~$A$. The algorithm given in~\cite{kauers15b}
for computing (global) integral bases computes local integral bases at finite
points as an intermediate step. The computation of a local integral basis at
infinity is also possible, either analogously, or by making a change of
variables $\tilde x=1/x$ and computing a local integral basis at~$0$.

An integral basis $\omega_1,\dots,\omega_n$ is always also a $K(x)$-vector space
basis of~$A$. A key feature of integral bases is that they make poles explicit. Writing
an element $f\in A$ as a linear combination $f=\sum_{i=1}^n f_i\omega_i$ for some
$f_i\in K(x)$, we have that $f$ has a pole at $a\in\bar C$ if and only if at least one
of the $f_i$ has a pole there.

\begin{lemma}\label{lemma:1}
  Let $L$ be a fuchsian operator and let
  $\omega_1,\dots,\omega_n$ be a local integral basis of $A=K(x)[\partial_x]/\<L>$ at $a\in\bar C\cup\{\infty\}$.
  Let $f\in A$ and $f_1,\dots,f_n\in K(x)$ be such that $f=\sum_{i=1}^nf_i\omega_i$.
  Then $f$ is integral at $a$ if and only if each $f_i\omega_i$ is integral at~$a$.
\end{lemma}
\begin{proof}
  The direction ``$\Rightarrow$'' is obvious. To show ``$\Leftarrow$'', suppose
  that $f$ is integral at~$a$. Then there exist $w_1,\dots,w_n\in\bar C(x)_a$ such that
  $f=\sum_{i=1}^nw_i\omega_i$. Thus $\sum_{i=1}^n(w_i-f_i)\omega_i=0$, and then
  $w_i=f_i$ for all $i$, because $\omega_1,\dots,\omega_n$ is a basis.
  As elements of $\bar C(x)_a$, the $f_i$ are integral at~$a$, and hence also all the $f_i\omega_i$
  are integral at~$a$.
\end{proof}

The lemma says in particular that poles of the $f_i$ in a linear combination
$\sum_{i=1}^n f_i\omega_i$ have no chance to cancel each other.

\begin{lemma}\label{lemma:e}
  Let $L$ be a fuchsian operator and let
  $\omega_1,\dots,\omega_n$ be an integral basis of $A=K(x)[\partial_x]/\<L>$.
  Let $e\in K[x]$ and
  $M=((m_{i,j}))_{i,j=1}^n\in K[x]^{n\times n}$ be such that
  \[
    e\,\omega_i'=\sum_{j=1}^n m_{i,j}\omega_j
  \]
  for $i=1,\dots,n$ and $\gcd(e,m_{1,1},\dots,m_{n,n})=1$.
  Then $e$ is square-free.
\end{lemma}
\begin{proof}
  Let $a$ be a root of~$e$. We show that $a$ is not a multiple root.
  Since $\omega_i$ is integral, it is in particular locally integral at~$a$.
  Therefore $(x-a)\omega_i'$ is locally integral at~$a$.
  Since $\omega_1,\dots,\omega_n$ is an integral basis, it follows that
  $(x-a)m_{i,j}/e\in\bar C(x)_a$ for all~$i,j$.
  Because of $\gcd(e,m_{1,1},\dots,m_{n,n})=1$, no factor $x-a$ of $e$
  can be canceled by all the~$m_{i,j}$.
  Therefore the factor $x-a$ can appear in $e$ only once.
\end{proof}

\begin{lemma} \label{lemma:degM}
  Let $L$ be a fuchsian operator and let $\omega_1,\dots,\omega_n$ be a local integral
  basis at infinity of $A=K(x)[\partial_x]/\<L>$. Let $e\in K[x]$ and $M=((m_{i,j}))_{i,j=1}^n\in K[x]^{n\times n}$
  be defined as in Lemma~\ref{lemma:e}. Then $\deg_x(m_{i,j})<\deg_x(e)$ for all $i,j$.
\end{lemma}
\begin{proof}
  Since every $\omega_i$ is local integral at infinity, so is every $x\,\omega_i'$.
  Since $\omega_1,\dots,\omega_n$ is an integral basis at infinity, it follows that
  $xm_{i,j}/e\in\bar C(x)_\infty$ for all~$i,j$. This means that $1+\deg_x(m_{i,j})\leq\deg_x(e)$
  for all~$i,j$, and therefore $\deg_x(m_{i,j})<\deg_x(e)$, as claimed.
\end{proof}

A $K(x)$-vector space basis $\omega_1,\dots,\omega_n$ of $A=K(x)[\partial_x]/\<L>$ is
called \emph{normal} at $a\in\bar C\cup\{\infty\}$ if there exist $r_1,\dots,r_n\in
K(x)$ such that $r_1\omega_1,\dots,r_n\omega_n$ is a local integral basis
at~$a$. Trager shows for the case of algebraic function fields how to construct
an integral basis which is normal at infinity from a given integral basis and
a given local integral basis at infinity~\cite{trager84}. The same procedure also applies
in the present situation. % It works as follows.

/// MK: I WILL ADD THIS IN THE END, IF SPACE PERMITS. OTHERWISE WE DROP IT.

Although normality is a somewhat weaker condition on a basis than integrality,
it also excludes the possibility that poles in the terms of a linear combination
of basis elements can cancel:

\begin{lemma}\label{lemma:3}
  Let $L$ be a fuchsian operator and let $\omega_1,\dots,\omega_n$ be an integral basis of $A=K(x)[\partial_x]/\<L>$
  which is normal at infinity.
  Let $f=\sum_{i=1}^n f_i\omega_i$ for some $f_1,\dots,f_n\in K[x]$.
  Then $f$ has a pole at infinity if and only if
  there is some $i$ such that $f_i\omega_i$ has a pole at infinity.
\end{lemma}
\begin{proof}
  Let $r_1,\dots,r_n\in K(x)$ be such that $r_1\omega_1,\dots,r_n\omega_n$ is a
  local integral basis at infinity. By $f=\sum_{i=1}^n
  (f_ir_i^{-1})(r_i\omega_i)$ and by Lemma~\ref{lemma:1}, $f$~is integral at infinity iff all
  $f_ir_i^{-1}r_i\omega_i=f_i\omega_i$ are integral at infinity.
\end{proof}

\section{Hermite Reduction}\label{sec:hermite}

Hermite reduction and its generalizations are the key step in many integration
algorithms, including the telescoping algorithm presented in this paper. It
turns out that the Hermite reduction for fuchsian D-finite functions is
literally the same as Trager's reduction for algebraic
functions~\cite{trager84}.

We start with a technical lemma, which is needed later to ensure that the
Hermite reduction always works. The analogous statement for algebraic
functions and its proof can be found in~\cite[pp. 46--47]{trager84}; for
sake of self-containedness, we give the full proof here, although it is a
straightforward generalization of Trager's proof.

Throughout this section, let $L\in K(x)[\partial_x]$ be a fuchsian operator
of order~$n$ and let $A=K(x)[\partial_x]/\<L>$.

\begin{lemma}\label{lemma:ibv}
Let $\{\omega_1,\ldots,\omega_n\}$ be an integral basis of~$A$.
Moreover, let $u,v\in K[x]$ be such that $uv\omega_i'$ is
integral for $1\leq i\leq n$ and $\gcd(u,v)=\gcd(v,v')=1$.  For some integer
$m>1$ we define $\psi_i:=uv^m(\omega_i/v^{m-1})'$; then
$\{\psi_1,\ldots,\psi_n\}$ is a local integral basis at each root $a\in\bar C$ of~$v$.
\end{lemma}
\begin{proof}
By expanding $\psi_i=uv\omega_i'-(m-1)uv'\omega_i$ one sees that the $\psi_i$
themselves are integral elements. We
have to show that each integral element can be expressed as a linear
combination of the $\psi_i$ with coefficients in $\bar C(x)_a$. To the contrary,
assume that there exists an integral element~$f$ that requires $x-a$ in the
denominator of some coefficient, i.e.,
\[
  f = \frac{1}{v} \sum_{i=1}^n c_i \psi_i \quad\text{with }c_i\in \bar{C}(x)_a
  \text{ and } c_i(a)\neq0 \text{ for some } i
\]
(here we use the fact that $v$ is squarefree).  Further let $g=\sum_{i=1}^n
(c_iu)'\omega_i$, which is obviously integral. Then also their sum
\begin{align*}
  f+g &= v^{m-1} \sum_{i=1}^n \left(c_iu \left(\frac{\omega_i}{v^{m-1}}\right)'
  + (c_iu)'\frac{\omega_i}{v^{m-1}} \right) \\
  &= v^{m-1} \sum_{i=1}^n \left(\frac{uc_i\omega_i}{v^{m-1}}\right)'
\end{align*}
must be integral. Since $\{\omega_1,\ldots,\omega_n\}$ is an integral basis,
there exists for each $i=1,\ldots,n$ a series solution $y_i\in C[[[x-a]]]$
of~$L$ such that $\omega_i\cdot y_i$ involves a term
$T=(x-a)^\alpha\log(x-a)^\beta$ with $0\leq\alpha<1,\beta=0$ or
$0<\alpha\leq1,\beta>0$. Let now $i$ be an index such that $c_i(a)\neq0$; together
with $\gcd(u,v)=1$ this implies that $T$ appears in $(uc_i\omega_i)\cdot y_i$.
Using the fact that the $\omega_i$ form an integral basis, it follows by
Lemma~\ref{lemma:1} that $T$ is also present in $h\cdot y_i$ where
$h=\sum_{i=1}^n uc_i\omega_i$. Applying the operator $v^{m-1}\partial_xv^{1-m}$
to $h\cdot y_i$ turns $T$ into $(x-a)^{\alpha-1}\log(x-a)^\beta$, which reveals that
$v^{m-1}(h/v^{m-1})'=f+g$ is not integral. This contradicts our assumption of
$f$ being integral, and hence $\{\psi_1,\ldots,\psi_n\}$ is a local integral
basis at~$a$.
\end{proof}

Let $\{\omega_1,\ldots,\omega_n\}$ be an integral basis for~$A$.
Further let $e,m_{i,j}\in K[x]$ ($1\leq i,j\leq n$) be such that
$e\omega_i'=\sum_{j=1}^n m_{i,j}\omega_i$ and
$\gcd(e,m_{1,1},m_{1,2},\ldots,m_{n,n})=1$ as in Lemma~\ref{lemma:e}. For describing
the Hermite reduction we fix an integrand $f\in A$ and represent it in the
integral basis, i.e., $f=\sum_{i=1}^n (f_i/d)\,\omega_i$ with
$d,f_1,\ldots,f_n\in K[x]$. The purpose is to find $g,h\in A$ such that
$f=g'+h$ and $h=\sum_{i=1}^n(h_i/d^\ast)\omega_i$ with $h_1,\ldots,h_n\in K[x]$
and $d^\ast$ denoting the squarefree part of~$d$.
As differentiating the $\omega_i$ can introduce
denominators, name\-ly the factors of~$e$, it is convenient to consider those
denominators from the very beginning on, and this means that we want to assume
$e\mid d$. Note that $\gcd(d,f_1,\ldots,f_n)$ may be nontrivial.

We now describe one step of the Hermite reduction, where the multiplicity
$m>1$ of some nontrivial squarefree factor~$v\in K[x]$ of $d$ is reduced.
Let $u\in K[x]$ be such that $d=uv^m$; it follows that $\gcd(u,v)=1$ and
$\gcd(v,v')=1$. We want to find $g_1,\ldots,g_n,h_1,\ldots,h_n\in K[x]$
such that
\begin{equation}\label{eq:hred}
  \sum_{i=1}^n \frac{f_i}{uv^m}\omega_i =
  \biggl(\sum_{i=1}^n\frac{g_i}{v^{m-1}}\omega_i\biggr)' +
  \sum_{i=1}^n \frac{h_i}{uv^{m-1}}\omega_i.
\end{equation}
By a repeated application of such reduction steps one can decompose any $f\in A$
as $f=g'+h$ where the denominators of the coefficients of $h$ are squarefree.
% This gives rise to the reduction map $[f]=h$.

In order to determine the unknown polynomials $g_1,\ldots,g_n$ in~\eqref{eq:hred},
the denominator $uv^m$ is cleared which yields
\begin{equation}\label{eq:clear}
  \sum_{i=1}^n f_i\omega_i = \sum_{i=1}^n \biggl( uvg_i'\omega_i +
  uv^mg_i\left(\frac{\omega_i}{v^{m-1}}\right)' + vh_i\omega_i \biggr),
\end{equation}
and then this equation is reduced modulo~$v$:
\begin{equation}\label{eq:modv}
  \sum_{i=1}^n f_i\omega_i =
  \sum_{i=1}^n uv^mg_i\left(\frac{\omega_i}{v^{m-1}}\right)' \mod v.
\end{equation}
By Lemma~\ref{lemma:ibv} the elements $uv^m(\omega_i/v^{m-1})'$ form a local
integral basis, which implies that the coefficients $g_i$ are uniquely
determined.

By Lemma~\ref{lemma:e} the polynomial~$e$ is squarefree and therefore $e\mid uv$;
hence we can write $uv=ew$ for some $w\in K[x]$. By rewriting the
derivatives of the $\omega_i$ in terms of the integral basis,
Equation~\eqref{eq:modv} turns into
\begin{align*}
  \sum_{i=1}^n f_i\omega_i
  &= \sum_{i=1}^n ug_i \bigl( v\omega_i' - (m-1)v'\omega_i \bigr) \mod v\\
  &= \sum_{i=1}^n g_i\, \biggl( w\sum_{j=1}^n m_{i,j}\omega_j - (m-1)uv'\omega_i \biggr) \mod v.
\end{align*}
Coefficient comparison with respect to $\omega_1,\ldots,\omega_n$ yields a
system of linear equations for $g_1,\ldots,g_n$. Note that $\deg_x(g_i)<\deg_x(v)$
since this system is solved modulo~$v$.

The remaining unknowns $h_1,\ldots,h_n$ are obtained by plugging
$g_1,\ldots,g_n$ into Equation~\eqref{eq:clear}.


\begin{example}
TODO
\end{example}

\section{The Canonical Form Property}

Recall from the introduction that reduction based creative telescoping requires
some $K$-linear reduction map $[\cdot]\colon A\to A$ with the property that
$f-[f]$ is integrable in $A$ for every $f\in A$. This is sufficient for the
correctness of the method, but additional properties are needed in order to
prove that the method terminates.

As also explained already in the introduction, one possibility consists in
showing that $[f]=0$ whenever $f$ is integrable. In the present section,
we will show that things can be arranged such that the Hermite reduction
has this property. The key is the following theorem, which was already shown
by Trager for algebraic functions.

\begin{theorem}\label{thm:intiff0}
  Suppose that $f\in A$ has a double root at infinity,
  i.e., that for every solution $y\in C[[[x^{-1}]]]$ of $L$
  the series $f\cdot y$ contains only monomials $x^\alpha\log(x)^\beta$ with $\alpha\leq-2$.
  Let $B=\{\omega_1,\ldots,\omega_n\}$ be an integral basis for~$A$
that is normal at infinity.
% Then the Hermite remainder $[f]$ w.r.t.~$B$
If $f=g'+h$ is the result of the Hermite reduction described above,
then $h$ is zero if and only if $f$ is integrable in~$A$.
\end{theorem}
\begin{proof}
The direction ``$\Rightarrow$'' is trivial. To show the implication
``$\Leftarrow$'' assume that $f$ is integrable. From $f=g'+h$ it follows that
also $h$ is integrable; let $H\in A$ be such that $H'=h$.  In order to show
that $h=0$, we show that $H$ is constant.  To show that $H$ is constant, we
show that it has neither finite poles nor a pole at infinity.

It is clear that $H$ has no finite poles because $h$ has only ``simple'' poles
(i.e., all series expansions of $h$ have only terms of the form
$(x-a)^{-1}$ or $(x-a)^\alpha\log(x-a)^\beta$ with $\alpha>-1$).

We will show below that when $H$ has a pole at infinity, then also $g+H$ must
have a pole at infinity.  Since $f=g'+h=(g+H)'$ has at least a double root at
infinity, $g+H$ has at least a single root at infinity. This excludes the
possibility that $H$ has a pole at infinity.

Assume now that $H$ has a pole at infinity. We show that $g+H$ has also a pole at infinity.
Since $H$ has no finite poles, $H$ is integral and we can write it as
$H=p_1\omega_1+\cdots+p_n\omega_n$ with $p_i\in C[x]$.
Since we assume that $H$ has a pole at infinity, it follows by Lemma~\ref{lemma:3} that
(a) one of the $p_i$ has positive degree or
(b) some non-constant $\omega_i$ has a nonzero coefficient~$p_i$.
From how the Hermite reduction works, we know that $g=\sum_i g_i\omega_i$
with proper rational functions~$g_i$. Consider now $g+H=\sum_i (g_i+p_i)\omega_i$.
In case (a) there is some $i$ for which $g_i+p_i$ has a pole at infinity,
and thus $(g_i+p_i)\omega_i$ has a pole at infinity (because $\omega_i$ has
no poles at finite places and therefore no zero at infinity).
In case (b), there is some $i$ with $p_i\neq0$ and $\omega_i$ is non-constant,
which implies that $(g_i+p_i)\omega_i$ has a pole at infinity.
In both cases, therefore, $g+H$ has a pole at infinity by Lemma~\ref{lemma:3}.
\end{proof}

Note that the condition in Theorem~\ref{thm:intiff0} that $f$ has a double
root at infinity is not a restriction at all, but it can always be achieved by
a suitable change of variables. Let $a\in C$ be a regular point of~$L$; by
the transformation $x\to a+1/x$ the regular point~$a$ is moved to
infinity. From
\[
  \int f(x) \,\mathrm{d}x = \int f\left(\frac{1}{x}+a\right)\left(-\frac{1}{x^2}\right) \mathrm{d}x
\]
we see that the new integrand has a double root at infinity.

Moreover, since the action of $\partial_t$ on series domains is defined coefficient-wise,
it follows that when $f$ has (at least) a doulbe root at infinity (with respect to~$x$), 
this is also true for $\partial_t\cdot f, \partial_t^2\cdot f, \partial_t^3\cdot f,\dots$,
and then also for every $K$-linear combination $p_0f+p_1\partial_tf+\cdots+p_r\partial_t^rf$.
This means that $p_0+p_1\partial_t+\cdots+p_r\partial_t^r$ is a telescoper for $f$ if
\emph{and only if} $[p_0+p_1\partial_t+\cdots+p_r\partial_t^r]=0$.

We already know for other reasons~\cite{Zeilberger1990,chyzak00} that
telescopers for D-finite functions exist, and therefore the reduction-based
creative telescoping procedure with Hermite reduction with respect to an
integral basis that is normal at infinity as reduction function succeeds when
applied to an integrand $f\in A$ that has a double root at infinity. Again, if
$f$ has no double root at infinity, we can produce one by a change of variables.
In particular, the method finds a telescoper of smallest possible order.

\section{Polynomial Reduction}\label{sec:polynomial}

The argument in the previous section based on Theorem~\ref{thm:intiff0}
has the advantage that it ensures that the algorithm will find the minimal telescoper.
A disadvantage is that it does not tell us in advance how large this telescoper will be.
Not even the existence of a telescoper follows from this approach, but has to be imported
into the argument from elsewhere.

An independent proof of the existence of telescopers could be obtained if we could show
that the $K$-vector space $\{\,[f]:f\in A\,\}$ has finite dimension. Note that this does
not follow from Theorem~\ref{thm:intiff0}. In fact, it is not true if $[\cdot]$ is just
the Hermite reduction. We therefore introduce below an additional reduction, called
\emph{polynomial reduction,} which we apply after Hermite reduction. We then show that
the combined reduction (Hermite reduction followed by polynomial reduction) has the
desired dimension property for the space of remainders. As a result, we obtain a new
proof for the bound on the order of the telescoper given in~\cite{chen14a}.

In this approach, we use two integral bases. First we use a global integral basis (not
necessarily normal at infinity) in order perfrom Hermite reduction. Then we write the
remainder $h$ with respect to some local integral basis at infinity and perform the
polynomial reduction on this representation.

The Hermite reduction decomposes the input element~$f\in A$ into the form
\[f = \partial_x(g) + h,\]
where~$g, h\in A$ and~$h = \sum_{i=1}^n \frac{h_i \omega_i}{d\cdot e}$ with~$h_i, d, e\in K[x]$ and~$\gcd(d, e)=1$.
Recall that $\{\omega_1, \ldots, \omega_n\}$ is an integral basis of~$A$ and
\[e \cdot  \partial_x(\omega_i) = \sum_{j=1}^n m_{i, j}\omega_j,\]
where~$e, m_{i, j}\in K[x]$ and~$\gcd(e, m_{1, 1}, m_{1, 2}, \ldots, m_{n, n})=1$.
By the extended Euclidean algorithm, we compute~$u_i, v_i\in K[x]$ such that
$\deg_x(v_i) < \deg_x(d)$ and
\[h_i = u_i d + v_i e.\]
Then we get
\[ \sum_{i=1}^n \frac{h_i\omega_i}{d \cdot e} =  \sum_{i=1}^n \left(\frac{u_i}{e} + \frac{v_i}{d}\right)\omega_i.\]
We now introduce the \emph{polynomial reduction} for confining the $u_i$'s to a finitely
dimensional vector space over~$K$. Similar reductions had been introduced and used in creative telescoping
for hyperexponential functions~\cite{bostan13a} and hypergeometric terms~\cite{chen15a}.

Let~$V = (\nu_1, \ldots, \nu_n)^T$ whose entries form a basis of~$A$ over~$K(x)$. Assume that
$a \partial_x(V) = BV$ with~$a\in K[x]$, $B = (b_{i, j})\in K[x]^{n \times n}$
and $\gcd(a, b_{1, 1}, b_{1, 2}, \ldots, b_{n ,n})=1$. Let~$p = (p_1, \ldots, p_n)\in K[x]^n$. Then
\begin{equation} \label{EQ:polyred}
\partial_x (p V)=\partial_x \sum_{i=1}^n p_i \nu_i = \frac{a\partial_x(p) + B\cdot p}{a}\, V.
\end{equation}
This motivates us to introduce the map~$\phi_V: K[x]^n \rightarrow K[x]^n$
defined by~$\phi_V(p) = a \partial_x(p) + B \cdot p$ for any~$p\in K[x]^n$. We call $\phi_V$
the \emph{map for polynomial reduction} with respect to~$V$, and call
the subspace $\im(\phi_V) = \{\phi_V(p) \mid p \in K[x]^n\}$
the \emph{subspace for polynomial reduction} with respect to~$V$.
We can always view the polynomial matrix~$B\in K[x]^{n\times n}$ as a polynomial in~$x$
with matrix coefficients of the form
\[B = B_{\deg_x(B)}x^{\deg_x(B)} + \ldots + B_0, \]
where~$B_i\in K^{n\times n}$.
Let~$\{e_1, \ldots, e_n\}$ be the standard basis of~$K^n$. Then the module $K[x]^n$ viewed as a vector space over~$K$
is then spanned by
\[\cS := \{e_ix^j \mid 1\leq i \leq n, j\in \bN\}.\]
For any~$p \in K[x]^n$, we can always write as
\[p = \vp_sx^s + \cdots + \vp_0,\]
where~$\vp_i\in K^n$ and~$\vp_s \neq 0$. We call $\vp_sx^s$
the \emph{leading term} of~$p$ in~$x$ and~$\vp_s$ the corresponding \emph{leading coefficient},
denoted by~$\lt_x(p)$ and~$\lc_x(p)$, respectively.
Let~$N_{\phi_V}$ be the subspace of~$K[x]^n$ spanned by
\[\{t \in \cS \mid t \neq \lt(p) ~\text{for all~$p\in \im(\phi_V)$}. \}\]
Then~$K[x]^n = \im(\phi_V) \oplus N_{\phi_V}$. We call~$N_{\phi_V}$ the \emph{standard complement} of~$\im(\phi_V)$.
\begin{prop}\label{PROP:finite}
If~$\deg_x(B) \leq \deg_x(a)-1$, then $N_{\phi_V}$ is a finitely dimensional
vector space over~$K$.
\end{prop}
\begin{proof} Let~$p\in K[x]^n$ with~$s=\deg_x(p)$ and~$\mu := \deg_x(B)$.
We first consider the case when $\mu <\deg_x(a)-1$.
If~$s=0$, then~$\phi_V(p) = B p$,
which implies that~$\deg_x(\phi_V(p))\leq \mu$. If~$s>0$, then
\[\phi_V(p)=\lc(a)s\lc(p)x^{s+\deg_x(a)-1} +~ \text{lower terms in~$x$}.\]
Therefore, any element~$q\in K[x]^n$ of degree at least~$\deg_x(a)$ can be written as
$q = \phi_V(q_1) + q_2$ for some~$q_1, q_2\in K[x]^n$ such that~$\deg_x(q_2)\leq \deg_x(a)-1$.
Thus, the dimension of $N_{\phi_V}$ is at most $n (\deg_x(a)-1)$.

Next we consider the case when~$\mu =\deg_x(a)-1$. Then
\[\phi_V(p) = (s\lc(a) + B_{\mu})\lc(p)x^{s+\mu}+\text{lower terms in~$x$}.\]
If~$-s\lc(a)\in K$ is not an eigenvalue of~$B_{\mu}\in K^{n\times n}$, then
the matrix~$J_s = s\lc(a) + B_{\mu}$ is invertible and then $e_ix^{s+\mu}$ is not in~$N_{\phi_V}$
for any~$i=1, \ldots, n$. Since~$B_\mu$ has at most~$n$ eigenvalues, the dimension of~$N_{\phi_V}$
is therefore at most~$n^2+n(\deg_x(a)-1)$. This completes the proof.
\end{proof}
\begin{remark}
{\red To explain how to find the basis of~$N_{{\phi_V}}$.}
\end{remark}

In general, the inequality~$\deg_x(M)\leq \deg_x(e)-1$ may not hold
for an integral basis~$\Omega := \{\omega_1, \ldots, \omega_n\}$ of~$A$. By Lemma~\ref{lemma:degM}, we could
change the basis into a local integral basis at the infinity. The simplest
way to achieve this is to make a change of variables~$\tilde x = 1/x$.
Let~$\tilde \omega_i(x) = \omega_i(1/x)$. Then $\tilde \Omega := \{\tilde \omega_1, \ldots, \tilde \omega_n\}$
is a local integral basis of~$A$ at any place except at the origin. If~$\Omega$
is normal at the infinity, then $\tilde \Omega$ is normal at the origin.
By $\partial_x \tilde \omega_i(x) = \partial_x(\omega_i)(1/x)\cdot \frac{-1}{x^2} $, we get
\[\partial_x \tilde \omega_i(x) = \sum_{i=1}^n\frac{-m_{i, j}(1/x)}{x^2e(1/x)}  \tilde \omega_j(x) = \sum_{i=1}^n \frac{\tilde m_{i, j}}{\tilde e}  \tilde \omega_j(x) .\]
If $\deg_x(M) \geq \deg_x(e)$, then~$\deg_x(\tilde M)<\deg_x(\tilde e)-1$.
{\red To add sentences about the transformation matrix between two basis... }

Since~$K[x]^n = \im(\phi_{\tilde \Omega}) \oplus N_{\phi_{\tilde \Omega}}$,
we can write any vector~$u = (u_1, \ldots, u_n)\in K[x]^n$
as $u = \phi_{\tilde \Omega}(p) + q$ with~$p\in K[x]^n$ and~$q\in N_{\phi_{\tilde \Omega}}$. Then
\[\sum_{i=1}^n \left(\frac{u_i}{\tilde e}\right){\tilde \omega}_i = \partial_x\left(\sum_{i=1}^n p_i {\tilde \omega}_i\right)
+ \sum_{i=1}^n \left(\frac{q_i}{\tilde e}\right){\tilde \omega}_i.\]
Combining the Hermite reduction and polynomial reduction, any element $f\in A$ can be
decomposed into the form
\[f = \partial_x(g) +  \sum_{i=1}^n \left(\frac{u_i}{\tilde e}\tilde \omega_i + \frac{v_i}{d}\omega_i\right),\]
where~$g\in A$ and $u_i, v_i\in K[x]$ with $u_i \in N_{\phi_{\tilde \Omega}}$ and $\deg_x(v_i) < \deg_x(d)$.
Moreover, $\tilde e$ and~$d$ only depend on the given integral basis~$\{\omega_1, \ldots, \omega_n\}$.

// TODO: DISSUSS IMPLICATIONS FOR CREATIVE TELESCOPING. BOUND. 

\section{Conclusion}

[[IF NECESSARY]]

\bibliographystyle{abbrv}
\bibliography{Hermite}

\end{document}
