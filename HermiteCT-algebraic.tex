\documentclass{sig-alternate}
\let\proof\undefined
\let\endproof\undefined
\usepackage{amsthm,amssymb,amsmath}
\usepackage{graphics}
\usepackage{bbm}
\usepackage{tikz}
\usepackage[plainpages=false,pdfpagelabels,colorlinks=true,citecolor=blue,hypertexnames=false]{hyperref}
\usepackage{color}
\overfullrule=1mm

\newtheorem{theorem}{Theorem}
\newtheorem{notation}[theorem]{Convention}
\newtheorem{prop}[theorem]{Proposition}
\newtheorem{corollary}[theorem]{Corollary}
\newtheorem{lemma}[theorem]{Lemma}
\newtheorem{remark}[theorem]{Remark}
\newtheorem{algorithm}[theorem]{Algorithm}
\newtheorem{problem}[theorem]{Problem}
\newtheorem{defi}[theorem]{Definition}
\newtheorem{example}[theorem]{Example}
\newtheorem{fact}[theorem]{Fact}
\def\qed{\quad\rule{1ex}{1ex}}

\newcommand{\red}{\color{red}}
\newcommand{\blue}{\color{blue}}
\newcommand{\bQ}{ {\mathbb Q}}
\newcommand{\bA}{ {\mathbb A}}
\newcommand{\bE}{ {\mathbb E}}
\newcommand{\bB}{ {\mathbb B}}
\newcommand{\cB}{ {\mathcal B}}
\newcommand{\bC}{ {\mathbb C}}
\newcommand{\bF}{ {\mathbb F}}
\newcommand{\bN}{ {\mathbb N}}
\newcommand{\bZ}{ {\mathbb Z}}
\newcommand{\bK}{ {\mathbb K}}
\newcommand{\cP}{ {\mathcal P}}
\newcommand{\cM}{ {\mathcal M}}
\newcommand{\cN}{ {\mathcal N}}
\newcommand{\cS}{ {\mathcal S}}
\newcommand{\den } {{\rm den}}
\newcommand{\num }{{\rm num}}
\newcommand{\de } {\delta}
\newcommand{\cE}{ {\mathcal E}}
\newcommand{\pa}{\partial}
\newcommand{\spanning}{\text{span}}

\overfullrule=1ex

\newcommand{\ve} {{\bf e}}
\newcommand{\vp} {{\bf p}}
\newcommand{\vq} {{\bf q }}
\newcommand{\vx} {{\bf x}}

\let\set\mathbbm
\def\lc{\operatorname{lc}}
\def\lt{\operatorname{lt}}
\def\im{\operatorname{Im}}
\def\lclm{\operatorname{lclm}}

\begin{document}

\title{Reduction~Based Creative~Telescoping for [ALGEBRAIC|FUCHSIAN D-FINITE]~Functions}

\numberofauthors{1}

\author{\medskip
Shaoshi Chen$^{1,2}$, \, Manuel Kauers$^{3}$, \, Christoph Koutschan$^{4}$ \\
\smallskip
       \affaddr{$^1$KLMM,\, AMSS, \,Chinese Academy of Sciences, Beijing, 100190, (China)}\\
       \smallskip
       \affaddr{$^2$SCG, Faculty of Mathematics, University of Waterloo, Ontario, N2L3G1, (Canada)}\\
              \smallskip
       \affaddr{$^3$Institute for Algebra, Johannes Kepler University, Altenberger Stra\ss e 69,
 A-4040 Linz, (Austria)}\\
        \smallskip
       \affaddr{$^4$RICAM, Austrian Academy of Sciences, Altenberger Stra\ss e 69, A-4040 Linz, (Austria)}\\
       \smallskip
      \email{schen@amss.ac.cn, manuel.kauers@jku.at}\\
      \email{christoph.koutschan@ricam.oeaw.ac.at}
%Preliminary notes
}




\maketitle
%
\begin{abstract}
  // TODO 
\end{abstract}


\category{I.1.2}{Computing Methodologies}{Symbolic and Algebraic Manipulation}[Algebraic Algorithms]

\terms{Algorithms, Theory}

\keywords{Algebraic function, Integral basis, Trager's Reduction, Telescoper}

%\bigskip

\section{Introduction}\label{SECT:intro}

The classical question in symbolic integration is whether the integral of
a given function can be written in ``closed form''. In its most restricted form,
the question is whether for a given function~$f$ belonging to some domain $D$
there exists another function~$g$, also belonging to~$D$, such that $f=g'$. For
example, if $D$ is the field of rational functions, then for $f=1/x^2$ we can
find $g=-1/x$, while for $f=1/x$ no suitable $g$ exists. When no $g$ exists
in~$D$, there are several other questions we may ask. One possibility is to ask
whether there is some extension~$E$ of $D$ such that in $E$ there exists some
$g$ with $g'=f$. For example, in the case of elementary functions, Liouville's
principle restricts the possible extensions~$E$, and algorithms have been
designed to construct these extensions whenever possible. Another possibility is
to ask whether for some modification $\tilde f\in D$ of~$f$ there exists a $g\in
D$ such that $\tilde f=g'$. Creative telescoping is a question of this
type. Here we are dealing with domains~$D$ containing functions in several
variables, say $x$ and~$t$, and the question is whether there is a linear
differential operator~$P$, nonzero and free of~$x$, such that there exists a
$g\in D$ with $P\cdot f=g'$, where $g'$ denotes the derivative of $g$ with
respect to~$x$. Typically, $g$~itself has the form $Q\cdot f$ for some operator
$Q$ (which may be zero and need not be free of~$x$). In this case, we call $P$
a telescoper for~$f$, and $Q$ a certificate for~$P$.

Creative telescoping is the backbone of definite integration. Readers not
familiar with this technique are refered to the literature~\cite{PWZbook1996,Zeilberger1990c,Zeilberger1991,Zeilberger1990,Koepf1998}
for motivation, theory, algorithms, implementations, and applications. There are
several ways to find telescopers for a given $f\in D$. In recent years, an
approach has become popular which has the feature that it can find a telescoper
without also constructing the corresponding certificate. This is interesting
because certificates tend to be much larger than telescopers, and in some
applications only the telescoper is of interest. This approach was first
formulated for rational functions $f\in C(t,x)$ in~\cite{BCCL2010} and later
generalized to rational functions in several variables~\cite{..}, to
hyperexponential functions~\cite{..} and, for the shift case, to hypergeometric
terms~\cite{..} and binomial sums~\cite{..}. In the present paper, we will extend
the approach to [ALGEBRAIC|FUCHSIAN D-FINITE] functions.

The basic principle of the general approach is as follows. Assume that the
$x$-constants $\mathrm{Const}_x(D)=\{\,c\in D:c'=0\,\}$ form a field, i.e., that $D$
is a vector space over the field of $x$-constants. Assume further that there is
some $\mathrm{Const}_x(D)$-linear map $[\cdot]\colon D\to D$ such that for every
$f\in D$ there exists a $g\in D$ with $f-[f]=g'$. Such a map is called a
\emph{reduction.} For example, in $D=C(t,x)$ Hermite reduction produces for
every $f\in D$ some $g$ such that $f-g'$ is either zero or a rational function
with a square-free denominator. In this case, we can take $[f]=f-g'$.
In order to find a telescoper, we can compute $[f]$, $[D_t\cdot f]$, $[D_t^2\cdot f]$, \dots,
until we find that they are linearly dependent over $\mathrm{Const}_x(D)$.
When we find a relation
$p_0[f] + \cdots + p_r[D_t^r\cdot f] = 0$,
then, by linearity,
$[p_0 f + \cdots + p_r D_t^r\cdot f] = 0$,
and then, by definition of $[\cdot]$, there exists a $g\in D$ such that $(p_0+\cdots + p_rD_t^r)\cdot f=g'$.
In other words, $P=p_0+\cdots + p_rD_t^r$ is a telescoper. 

There are two ways to guarantee that this method works. The first consists in
showing that $\{\,[f]:f\in D\,\}$ is a finite dimensional vector space over
$\mathrm{Const}_x(D)$. This approach was taken in~\cite{..,..,..}. It has the
nice additional feature that every bound for the dimension of this vector space
gives rise to a bound for the order of the telescoper. In particular, it implies
the existence of a telescoper. The second way requires that we already know for
other reasons that a telescoper exists. The idea is then to show that the
reduction $[\cdot]$ has the property that when $f\in D$ is such that there
exists a $g\in D$ with $g'=f$, then $[f]=0$. If this is the case and
$P=p_0+\cdots+p_rD_t^r$ is a telescoper for~$f$, then $P\cdot f$ is integrable
in~$D$, so $[P\cdot f]=0$, and by linearity $[f]$, \dots, $[D_t^r\cdot f]$ are
linearly dependent over $\mathrm{Const}_x(D)$. This means that the method won't
miss any telescoper. In particular, this argument has the nice feature that we
are guaranteed to find a telescoper of smallest possible order~$r$. This
approach was taken in~\cite{..}.

Using an analog of Trager's Hermite reduction for algebraic
functions~\cite{..,..}  adapted to fuchsian D-finite functions, we provide an
algorithm for finding the minimal order telescoper in Section~\ref{..} using the
first argument. We then proceed to describe in Section~\ref{..} an additional
reduction which in combination with the Hermite reduction ensures that the
remainders live in a finite dimensional vector space [IF THE REDUCTION
  APPLIES]. This gives a new proof of a bound for the order of the telescopers,
and in particular an independent proof for their existence.

\section{Fuchsian D-finite Functions}

define operators, ``algebra'', series solutions at a point, fuchsianity.

comparison to algebraic function fields.

bivariate functions: we assume to have a main variable $x$ and a secondary variable~$t$.
main variable means that notions depending on series expansions refer to the expansion
wrt $x$ with coefficients depending on~$t$.

the ``algebra'' should be closed under differentiation wrt. both $x$ and~$t$. for algebraic
functions, one implies the other, but for d-finite functions, this is not automatically true.

\section{Integral Bases}

recall definition of valuation, integral series, integral ``algebra'' element, integral closure, integral basis.

note: differentiation with respect to $t$ cannot increase (decrease?) the valuation with respect to $x$
(meaning: poles at either a finite or the infinite place cannot become worse when differentiating wrt.~$t$)

lemma: main property about integral bases.

comparison to algebraic function fields.

define normal at infinity. point out that every integral basis can be made normal at infinity.
(same argument as in the algebraic case.)

lemma: main property about normal bases. 

\section{Hermite Reduction}

work out trager's hermite reduction for the fuchsian d-finite case. (actually, it is
literally the same as in the algebraic case.)

make no assumption on the regularity at infinity yet, because it is not needed for the
reduction but only for the theorem. also the assumption about ``normal at infinity'' is
not needed for the algorithm.

theorem: if $f$ has a double root at infinity then the hermite remainder wrt an integral
basis that is normal at infinity is zero iff $f$ is integrable in the ``algebra''.

remark: double root at infinity can always be achieved by a change of variables.
double root persists by differentiating wrt. $t$. consequence: creative telescoping works.
(note: it is known that telescopers exist~\cite{..,..,..}). 
we find the minimal telescoper.

\section{Polynomial Reduction}

second approach: we wish to show that the hermite remainders belong to a finite dimensional
vector space. in fact, this is not true. 
note that the theorem does not imply that the hermite remainder is uniquely determined by
$f$ and the basis (it is not). it only makes a statement about the integrable case. 

to make it true, we need to normalize the hermite remainder further. in view of the result
of the previous section, investing additional computational seems pointless, because we have
already shown that repeated hermite reduction suffices to detect the minimal order telescoper.
in fact, the main point of the algorithm we discuss next is that it gives rise to a bound on
the telescoper. we will obtain an independent proof of the bound that already appeared in \cite{GAZ}. 

===== shaoshi's text ahead =====

Let~$\bK$ be an algebraically closed field of characteristic zero.
Assume that $\alpha$ is an algebraic function over the rational function
field~$\bK(x)$ and $P\in \bK[x, y]$ be the corresponding minimal polynomial for~$\alpha$.
Let~$\bF = \bK(x, \alpha)$ be the algebraic extension of~$\bK(x)$ generated by~$\alpha$.
Assume that~$\Omega:=\{\omega_1, \ldots, \omega_n\}$ be an integral basis of~$\bF$
over~$\bK[x]$, where~$n=\deg_y(P)$. Let~$D_x=\frac{d}{dx}$ be the usual derivation on~$\bF$
with respect to~$x$. Assume that
\[D_x(\omega_i) = \frac{1}{Q}\sum_{j=1}^n m_{i, j}\omega_j,\]
where~$Q, m_{i, j}\in \bK[x]$ such that~$Q$ is coprime to the gcd of~$m_{i,j}$'s. Set~$M=(m_{i, j})\in \bK[x]^{n\times n}$
and
\[\deg_x(M):= \max\{\deg_x(m_{i, j})\}\].
\begin{lemma}\label{LEM:M}
$Q$ is squarefree and~$\deg_x(M) <= \deg_x(Q)-1$.
\end{lemma}
\begin{proof}{\red To be proved...}
\end{proof}

Since~$\Omega$ is an integral basis of~$\bF$, we can write~$\alpha\in \bF$ as
\[\alpha = \sum_{i=1}^n \frac{a_i\omega_i}{D}.\]
Applying Trager's reduction to~$\alpha$, we get
\[\alpha = D_x(\beta) + \sum_{i=1}^n \frac{c_i\omega_i}{D^*Q},\]
where~$\beta\in \bF$, $D^*, Q, c_i\in \bK[x]$, $\gcd(D^*, Q)=1$ and~$D^*$ is squarefree.
By the extended Euclidean algorithm, we compute~$u_i, v_i\in \bK[x]$ such that
$\deg_x(v_i) < \deg_x(D^*)$ and
\[c_i = u_i D^* + v_i Q.\]
Then we get
\[ \sum_{i=1}^n \frac{c_i\omega_i}{D^*Q} =  \sum_{i=1}^n \left(\frac{u_i}{Q} + \frac{v_i}{D^*}\right)\omega_i.\]
In order to control the number of supporting monomials of~$u_i$'s, we now introduce the polynomial reduction.

First, we rewrite the polynomial matrix~$M\in \bK[x]^{n\times n}$ as a polynomial in~$x$
with matrix coefficients of the form
\[M = m_{d-1}x^{d-1} + \ldots + m_0, \]
where~$m_i\in \bK^{n\times n}$, and~$d = \deg_x(Q)$. Define the map:
\[\phi: \bK[x]^n \rightarrow \bK[x]^n\]
by~$\phi(V)= QD_x(V) + MV$ for any~$V\in \bK[x]^n$. This map is $K[x]$-linear.
The image~$\im(\phi)$ is a subspace of~$K[x]^n$ as a vector space over~$\bK$.
Let~$\{\ve_1, \ldots, \ve_n\}$ be the standard basis of~$\bK^n$. Then the $\bK$-vector space~$\bK[x]^n$
is spanned by
\[\cS := \{\ve_ix^j \mid 1\leq i \leq n, j\in \bN\}.\]
For any~$p \in \bK[x]^n$, we can always write as
\[p = \vp_sx^s + \cdots + \vp_0,\]
where~$\vp_i\in \bK^n$ and~$\vp_s \neq 0$. We call $\vp_sx^s$
the \emph{leading term} of~$p$ in~$x$ and~$\vp_s$ the corresponding \emph{leading coefficient},
denoted by~$\lt_x(p)$ and~$\lc_x(p)$, respectively.
Let~$N_\phi$ be the subspace of~$K[x]^n$ spanned by
\[\{t \in \cS \mid s\neq \lt(p) ~\text{for all~$p\in \im(\phi)$}. \}\]
Then~$K[x]^n = \im(\phi) \oplus N_\phi$.
We call~$N_{\phi}$ the \emph{standard complement} of~$\im(\phi)$.
In the next, we will estimate the dimension of~$N_\phi$ as a vector space over~$\bK$.

{\red  {\bf This part is to be revised!!!} \noindent {\it Case 1.}~$\deg_x(M)<\deg_x(Q)-1$. Then for any~$p\in \bK[x]^n$ with~$s=\deg_x(p)$, we have
\[\phi(p) = s\lc(Q)\lc(p)x^{s+d-1}+\text{lower terms in~$x$}.\]
Then~$N_\phi$ is spanned by the set
\[\{\ve_ix^j \mid 1\leq i \leq n, 0\leq j \leq d-1\}.\]
Therefore, the dimension of~$N_{\phi}$ over~$\bK$ is~$nd$.

\noindent {\it Case 2.}~$\deg_x(M)=\deg_x(Q)-1$. Then for any~$p\in \bK[x]^n$ with~$s=\deg_x(p)$, we have
\[\phi(p) = (s\lc(Q) + m_{d-1})\lc(p)x^{s+d-1}+\text{lower terms in~$x$}.\]
If~$-s\lc(Q)\in \bK$ is not an eigenvalue of~$m_{d-1}\in \bK^{n\times n}$, then
the matrix~$J_s = s\lc(Q) + m_{d-1}$ is invertible and then $\ve_ix^{s+d-1}$ is not in~$N_\phi$
for any~$i=1, \ldots, n$. Let~$s_1, \ldots, s_\rho$ be distinct nonnegative integers such that
for each~$i=1, \ldots, \rho$, $-s_i\lc(Q)$ is an eigenvalue of~$m_{d-1}$. Without lose of generality,
we assume that~$s_1 <s_2<\ldots<s_\rho$. It is clear that~$\rho \leq n$
and the dimension of~$N_{\phi}$ over~$\bK$ is at most~$\rho n$.

{\bf To be continued...}}

\section{Conclusion}

[[IF DESIRED]]

\bibliographystyle{abbrv}
\bibliography{Hermite}

\end{document}
