\documentclass[a4paper,draft]{article}
\usepackage{amsmath}
\usepackage{amssymb}
\usepackage{bbm}
\usepackage{a4wide}

\usepackage{color}


\newcommand{\red}{\color{red}}

\parindent=0pt
\parskip\smallskipamount

\let\set\mathbbm
\def\<#1>{\langle#1\rangle}
\def\i{\mathrm{i}}
\def\e{\mathrm{e}}
\let\ideal\unlhd

\newcommand\todo[1][.]{\edef\tmpa{.}\edef\tmpb{#1}%
  \ifx\tmpa\tmpb
    \typeout{To Be on page \thepage}\fbox{\bf To Be}
  \else
    \typeout{To Be on page \thepage: #1}\fbox{{\bf To Be:} #1}
  \fi
}

\begin{document}

Working notes: Kauers, Koutschan $\to$ Chen\hfill 2015-12-09

\section{Setup}

$C$ a field, $K=C(x)$, $A=K(\alpha)=K[Y]/\<m>$ for some $m\in K[Y]$ absolutely irreducible.

\section{Preliminaries}

An element $f\in A$ is called integral at $\xi\in C\cup\{\infty\}$ if all
Puiseux series expansions of $f$ at places over $\xi$ have nonnegative
valuation.

For a Puiseux series $\sum_{n=n_0}^\infty a_n(x-\xi)^{n/r}$ with $a_{n_0}\neq0$
and $\xi\in C$, we say that $n_0/r$ is its valuation. A Puiseux series at
$\infty$ has the form $\sum_{n=n_0}^\infty a_n x^{-n/r}$ with $a_{n_0}\neq0$.
Its valuation is defined as~$n_0/r$.

To have negative valuation at some place (finite or infinite) means that the
corresponding ``function'' tends to infinity in a neighborhood of the place.

Let $B=\{w_1,\dots,w_n\}$ be a $K$-vector space basis of $A$.
\begin{enumerate}
\item $B$ is called a (local) integral basis
at $\xi\in C\cup\{\infty\}$ if $w_1,\dots,w_n$ are integral at $\xi$ and every
$f\in A$ which is integral at $\xi$ can be written as $f=p_1w_1+\cdots+p_nw_n$
for some $p_i\in C(x)$ whose valuation at $\xi$ is nonnegative. (For $\xi\in C$
this means that the denominators of the $p_i$ do not vanish at~$\xi$; for $\xi=\infty$
it means that $p_i(0)\neq0$ for all~$i$.)
\item $B$ is called a (global) integral basis
if it is a local integral basis at all $\xi\in C$. This is the case if and only if
(a) every $w_i$ is integral at all $\xi\in C$, and
(b) every element $f\in A$ which is integral at all $\xi\in C$ can be written
as $f=p_1w_1+\cdots+p_nw_n$ for some $p_i\in C[x]$.
\item $B$ is called a (local) normal basis at $\xi\in C\cup\{\infty\}$ if there
  exist $r_1,\dots,r_n\in C(x)=K$ such that $\{r_1w_1,\dots,r_nw_n\}$ is a local
  integral basis at~$\xi$.
\end{enumerate}

We know how to find an integral basis. Trager (pp.~23f) explains how to transform an integral
basis into an integral basis which at the same time is a local normal basis at~$\infty$.
His technique seems to generalize to the fuchsian D-finite case.

MAIN PROPERTY FOR NORMAL AT INFINITY: $\sum_i b_iw_i$ as a pole at infinity as soon as one term $b_iw_i$ has.

\section{Trager's Hermite Reduction}

Let $f\in A$ be given. We want to decide whether there exists $g\in A$ such that $g'=f$.
Hermite reduction produces $g,h\in A$ such that $f=g'+h$ and $h$ is in some sense as small
as possible. We call $h$ a Hermite remainder of~$f$.

Let $w_1,\dots,w_n$ be an integral basis of~$A$. Write $f=\frac1d\sum_i a_iw_i$
for $d,a_1,\dots,a_n\in C[x]$.
Hermite reduction finds $g,h\in A$ such that $f=g'+h$ and $h=\frac1{d^*}\sum_i c_iw_i$,
where $d^*$ is the square free part of~$d$.

\section{Main Theorem}

\textit{%
  \textbf{Claim:}
  If $f$ has at least a double root at $\infty$
  (meaning no Puiseux series expansion of $f$ at infinity has a term $x^\alpha$ with $\alpha>-2$)
  and the integral basis $w_1,\dots,w_n$ is normal at infinity,
  then $h=0$ if and only if $f$ is integrable in~$A$.
}

\textbf{Proof.} (Trager, pp. 48ff)
``$\Rightarrow$'' is obvious, $g$ is the desired integral.

``$\Leftarrow$''. Assume that $f$ is integrable. Then also $h$ is integrable.
Let $H=\int h$.
In order to show that $h=0$, we show that $H$ is constant.
To show that $H$ is constant, we show that it has neither finite poles nor a pole at infinity.

It is clear that $H$ has no finite poles because $h$ has only ``simple'' poles (i.e., all Puiseux
expansions of $h$ have only exponents $\geq-1$).

We will show below that when $H$ has a pole at infinity, then also $g+H$ must have a pole at infinity.
Since $f=g'+h=(g+H)'$ has at least a double root at infinity, $g+H$ has at least
a single root at infinity. This excludes the possibility that $H$ has a pole at infinity.

Assume now that $H$ has a pole at infinity. We show that $g+H$ has also a pole at infinity.
Since $H$ has no finite poles, $H$ is integral and we can write it as
$H=p_1w_1+\cdots+p_nw_n$ with $p_i\in C[x]$.
Since we assume that $H$ has a pole at infinity, it follows that (a) one of the $p_i$ has positive degree
or (b) some nonconstant $w_i$ has a nonzero coefficient~$p_i$.
(TRAGER USES HERE THAT EVERY NONCONSTANT $w_i$ HAS A POLE AT INFINITY. THIS IS PLAUSIBLE, BUT
WHY IS THIS NECESSARY?)
We know that $g=\sum_i b_iw_i$ with proper rational functions~$b_i$ (by how the algorithm works).
Consider $g+H=\sum_i (b_i+p_i)w_i$.
In case (a) there is some $i$ for which $b_i+p_i$ has a pole at infinity, then $(b_i+p_i)w_i$ has a pole
at infinity (because $w_i$ has no poles at finite places and therefore no zero at infinity),
and in case (b), there is some $i$ with $p_i\neq 0$ and $w_i$ is non-constant, then $(b_i+p_i)w_i$ has a
pole at infinity.
In both cases, therefore, $g+H$ has a pole at infinity because the basis is normal
at infinity (``MAIN PROPERTY'').
This completes the proof that $g+H$ has a pole at infinity. \rule{1ex}{1ex}



\section{Polynomial reductions}
{\red In this section, we introduce the polynomial reduction in order to turn the form~$h$ computed by the
Hermite reduction into a residual form. The idea will be quite similar to that in the hyperexponential case.


For any~$f \in K(\alpha(x))$, Trager's reduction returns the decomposition
\[f  = g' + r, \quad \text{where~$r =\sum_{i=1}^n \frac{c_iw_i}{d^*}$ with~$c_i\in C[x]$ and~$d^*$ squarefree}.\]
Assume that $w_i' = \frac{1}{t}\sum_{j=1}^n s_{i, j}w_j$. Then we have
\[(\sum_{i=1}^n p_i w_i)' = \sum_{i=1}^n (p_i' + \frac{1}{t}\sum_{j=1}^n p_js_{i, j})w_i.\]
We can always assume that~$t\mid d^*$. Set~$d^* = ut$, and then~$\gcd(u, t)=1$, since~both $d^*$ and~$t$ are squarefree.
Then applying the extended Euclidean algorithm to~$c_i$ yields
\[c_i = a_iu + b_i t \quad \text{with~$\deg_x(b_i) < \deg_x(u)$}.\]
Therefore,
\[r =\sum_{i=1}^n \frac{c_iw_i}{d^*} = \sum_{i=1}^n \left(\frac{a_i}{t}+ \frac{b_i}{u}\right)w_i. \]
We now try to control the degree of~$a_i$'s. Given polynomials~$t, s_1, \ldots, s_n \in C[x]$,
define the  map $\phi: C[x]^n \rightarrow C[x]$ by
\[\phi((p_1, p_2, \ldots, p_n)) = tp_1' + s_1p_1+ s_2p_2 + \cdots + s_np_n.\]
Let~$M = \text{Im}(\phi)$ be the image of~$\phi$, which is a subspace of~$C[x]$ as a linear space over~$C$.
Then we have the decomposition~$C[x] = M \oplus N$, where~$N$ is the complementary subspace of~$\phi$.
For any polynomial~$p\in C[x]$, we can decompose~$p= p_M + p_N$ such that~$p_M\in M$ and~$p_N \in N$.
We now claim that the dimension of~$N$ is finite. Indeed, if any of $s_2, \ldots, s_n$ is nonzero, say~$s_2$, then
any polynomial $p\in C[x]$ can be written as $p = qs_2 + \tilde{p}$ with~$\deg_x(\tilde p) < \deg_x(s_2)$.
Note that~$qs_2 \in M$, and then the dimension of $N$ is at most~$\deg_x(s_2)$. Now suppose $s_2=\cdots=s_n=0$.
In this case, the map~$\phi$ is similar to that in the hyperexponential case (see Section 4.1 of the paper
[Hermite Reduction and Creative Telescoping for Hyperexponential Functions]). Then the dimension of $N$ is finite.

We now let~$\phi_i$ be the previous map with respect to ${t, s_{i, 1}, \ldots, s_{i, n}}$. Then we can control the degree of $a_i$'s by decomposing
\[a_i = \tilde a_i + \bar a_i, \]
where~$\tilde a_i \in \text{Im}(\phi_i)$ and~$\bar a_i$ is in the finitely dimensional complementary subspace.

Combining the Hermite reduction and the polynomial reduction, we now confine any algebraic function
into a finite-dimensional space.

In the hyperexponential or hypergeometric case, the assumption of differential reduced or shift reduced anables
us to show that the map $\phi$ is injective. But I have no idea how to show this property in the algebraic case.
This is needed to show the following claim.

{Claim: Let~$f\in K(\alpha)$. Applying the Hermite reduction and the polynomial reduction to $f$ yields $f = g' + r$.
Then $f= h'$ for some algebraic function~$h$ if and only if $r=0$. }











 }




\end{document}
